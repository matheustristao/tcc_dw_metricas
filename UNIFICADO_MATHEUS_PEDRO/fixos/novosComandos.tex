\newcommand{\curso}[1]{\def\imprimircurso{#1}}

\newcommand{\palavraChaveUm}[1]{\def\imprimirpalavrachaveum{#1}}
\newcommand{\palavraChaveDois}[1]{\def\imprimirpalavrachavedois{#1}}

\newcommand{\cdu}[1]{\def\nomecdu{#1}}
\newcommand{\dataDaAprovacao}[1]{\def\imprimirdatadaaprovacao{#1}}

\newcommand{\membroConvidadoUm}[1]{\def\imprimirmembroconvidadoum{#1}}
\newcommand{\membroConvidadoDois}[1]{\def\imprimirmembroconvidadodois{#1}}

\newcommand\BackgroundPic{%
	\put(0,0){%
		\parbox[b][\paperheight]{\paperwidth}{%
			\vfill
			\centering
			\includegraphics[width=\paperwidth,height=\paperheight,%
				keepaspectratio]{figuras/capa.eps}%
			\vfill
		}
	}
}

\newcommand{\cmark}{\ding{51}}%
\newcommand{\xmark}{\ding{55}}%
\renewcommand{\lstlistingname}{Código-Fonte}


  \lstdefinelanguage{JavaScript}{
  keywords={typeof, new, true, false, catch, function, return, null, catch, switch, var, if, in, while, do, else, case, break},
  keywordstyle=\color{blue}\bfseries,
  ndkeywords={class, export, boolean, throw, implements, import, this},
  ndkeywordstyle=\color{darkgray}\bfseries,
  identifierstyle=\color{black},
  sensitive=false,
  comment=[l]{//},
  morecomment=[s]{/*}{*/},
  commentstyle=\color{purple}\ttfamily,
  stringstyle=\color{red}\ttfamily,
  morestring=[b]',
  morestring=[b]"
}

\lstset{
   language=JavaScript,
   extendedchars=true,
   basicstyle=\footnotesize\ttfamily,
   showstringspaces=false,
   showspaces=false,
   numbers=left,
   numberstyle=\footnotesize,
   numbersep=9pt,
   tabsize=2,
   breaklines=true,
   showtabs=false,
   captionpos=b
}

\renewcommand{\imprimircapa}{%
  \begin{capa}%
    \center
	\AddToShipoutPicture*{\BackgroundPic}


    \vspace*{2.6in}
	{\textbf{\large\imprimirinstituicao}}
	\par
	{\textbf{\large\imprimircurso}}

	\vspace{0.5in}

    {\ABNTEXchapterfont\bfseries\LARGE\imprimirtitulo}
    \vspace*{\fill}
    
	\begin{flushright}
    	\textbf{{\large{Autores:\ \ \ Matheus Oliveira Tristão dos Anjos}}}
		\par
		\textbf{{\large{              Pedro da Cunha Tomioka\ \ \ \  }}}
		\par
    	\textbf{{\large{Orientador: \imprimirorientador}}}
    	\par

	\end{flushright}
		
    \vspace*{0.2in}
    \textbf{{\large\imprimirlocal}}
    \par
    \textbf{{\large\imprimirdata}}
    
    \vspace*{2.2in}
  \end{capa}

}

