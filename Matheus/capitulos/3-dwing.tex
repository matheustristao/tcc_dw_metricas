\chapter{\textit{Data Warehouse}} 


\textit{Data Warehouse} é uma base de dados que armazena suas informações de maneira orientada a satisfazer solicitações de tomadas de decisão \cite{chaudhuri1997}. A diferença entre um típico banco de dados transacional e um  \textit{Data Warehouse}, porém, consiste na maneira como esses dados são armazenados. Em vez de existirem múltiplos ambientes de decisão operando de forma independente, o que com frequência traz informações conflituosas, um \textit{Data Warehouse} unifica as fontes de informações relevantes, de maneira que a integridade  qualidade dos dados são garantidas. \cite{neeraj_sharma_2011}. Dessa forma,\citeonline{chaudhuri1997} afirma que o ambiente de \textit{Data Warehousing} possibilita que seu usuário realize buscas complexas de maneira mais amigável diretamente em um só ambiente, em vez de acessar informações através de relatórios gerados por especialistas. 

\section{Arquitetura de um ambiente de \textit{Data Warehousing}}

\citeonline{Inmon2002} descreve que o \textit{Data Warehouse} é uma coleção de dados que tem como característica ser orientada a assunto, integrada, não volátil e temporal. Por dados orientados a assunto, podemos entender que... . O fato do ambiente ser integrado remete ao fato dele ser alimentado com dados que têm como origem de múltiplas fontes, integrando esses dados de maneira a construir uma única orientação. Como um conjunto não volátil e temporal de dados, é entendido que a informação carregada remete a um determinado momento da aplicação, possibilitando assim acesso a diferentes intervalos de tempo, não havendo como modificá-los.

Para alcançar as características descritas, o ambiente de Data Warehousing segue o EXPLICAR ETL, tendo sua arquittura modelada da seguinte forma:

MOSTRAR FIGURA SOBRE A ARQUITETURA DW



Kimball2002