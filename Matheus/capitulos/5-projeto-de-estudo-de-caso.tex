\chapter{Projeto de estudo de Caso}

Esse capítulo irá tratar da estratégia de pesquisa adotada durante o trabalho, buscando estar de acordo com ...


\section{Definição sobre estudo de caso}

O estudo de caso é uma estratégia de pesquisa utilizada para investigar um tópico de manira empírica através de um conjunto de procedimentos pré-especificados \cite{yin2001estudo}. Buscando diferenciar o estudo de caso de outras estratégias de pesquisa, \citeonline{yin2001estudo} esclarece que um estudo de caso deve focalizar acontecimentos contemporâneos, não havendo assim exigência quanto ao controle sobre os eventos comportamentais. Dessa forma, o estudo de caso difere de um experimento pelo motivo que neste há controle e manipulação sobre os eventos, diferentemente do estudo de caso, que não os manipula. Em suma, \citeonline{schramm_notes_1971} define que a essência de qualquer estudo de caso reside em esclarecer uma decisão ou um conjunto de decisões, considerando o motivo pelo qual elas foram tomadas e qual os resultados das suas implementações \cite{schramm_notes_1971}. 

Buscando maior entendimento a respeito do estudo de caso proposto por esse trabalho, foram criadas algumas perguntas que são fundamentais para o seu entendimento:

\begin{easylist}[itemize]	
	
	& Qual o problema a ser tratado?
	& Qual a questão de pesquisa relacionada a esse problema?
	& Quais são os objetivos a serem alcançados nessa pesquisa?	
	& Como foi a seleção do estudo de caso?
	& Qual fonte dos dados coletados nessa pesquisa?
	& Qual o método de coleta de dados?
	
	\end{easylist}	
	
As perguntas acima serão respondidas nas próximas seções, de modo que o estudo de caso possa ser compreendido como um projeto de pesquisa e então ser executado

\subsection{Problema}

O PROBLEMA


\subsection{Questão de Pesquisa}

Segundo \citeonline{caldiera_goal_1994}, a questão de pesquisa deve ser capaz de caracterizar o objeto que está sendo medido, seja ele produto, processo ou recurso. Sob essa lógica, a seguinte questão de pesquisa foi criada após análise do problema:

( ESCREVER QUESTÃO DE PESQUISA)

Para atender a questão de pesquisa foi utilizado o mecanismo goal-question-metrics (GQM), usado para definir e interpretar um software operacional e mensurável. O GQM combina em si muitas das técnicas de medição e as generaliza para incorporar processos, produtos ou recursos, o que torna seu uso adaptável a ambientes diferentes \cite{caldiera_goal_1994}. 


\textbf{Objetivo 01:} (ESCREVER)

\textbf{Questão específica 01:} (ESCREVER)

\textbf{Fonte:} (ESCREVER)

\textbf{Métrica:} (ESCREVER)

\textbf{Questão específica 02:} (ESCREVER)

\textbf{Fonte:} (ESCREVER)

\textbf{Métrica:} (ESCREVER)

\textbf{Objetivo 02:} (ESCREVER)

\textbf{Questão específica 03:} (ESCREVER)

\textbf{Fonte:} (ESCREVER)

\textbf{Métrica:} (ESCREVER)


\textbf{Questão específica 04:} (ESCREVER)

\textbf{Fonte:} (ESCREVER)

\textbf{Métrica:} (ESCREVER)

\section{\textit{Background}}

Soluçao do Baufaker + alguma o que tiver

\section{Seleção}

Os dados foram coletados no TCU porque...

\section{Fonte dos dados coletados}

O dados foram coletados via análise do código fonte e questionários que...

\section{Ameaças a validade do estudo de caso}

\begin{easylist}[itemize]	

& \textbf{Validade do Constructo: } A validade de construção está presente na fase de coleta de dados, quando deve ser evidenciado as múltiplas fontes de evidência e a coleta de um conjunto de métricas para que se possa saber exatamente o que medir e quais dados são relevantes para o estudo, de forma a responder as questões de pesquisa \cite{yin2001estudo}. Buscou-se garantir a validade de construção ao definir objetivos com evidências diferentes. Estas, por sua vez, estão diretamente relacionadas com os objetivos do estudo de caso e os objetivos do trabalho. 

& \textbf{Validade interna: } Para \cite{yin2001estudo} o uso de várias fontes de dados e métodos de coleta permite a triangulação, uma técnica para confirmar se os resultados de diversas fontes e de diversos métodos convergem. Dessa forma é possível aumentar a validade interna do estudo e aumentar a força das conclusões.
A triangulação de dados se deu pelo  resultado da solução de DW que utiliza o código-fonte e foi explicada no capitulo X, de base de documentos, de questionários e de entrevistas para coleta de dados. A triangulação de métodos ocorreu pelo uso de métodos de coleta quantitativos e qualitativos

& \textbf{Validade externa: } Por este ser um caso único a generalização do estudo de caso se dá de maneira pobre (YIN, 2010), assim é necessário a utilização do estudo em múltiplos casos para que se comprove a generalidade dos resultados.
Como este trabalho é o primeiro a utilizar a solução para o estudo de caso no órgão, não há como correlacionar os resultados obtidos a nenhum outro estudo.
PERGUNTA: Os outros TCCs podem ser considerados outros casos e assim correlacioná-los a este TCC? 

& \textbf{Confiabilidade: } Com relação a confiabilidade, Yin associa à repetibilidade, desde que seja usada a mesma fonte de dados. Nesse trabalho o protocolo de estudo de caso apresentado nessa seção garante a repetibilidade desse trabalho e consequentemente a validade relacionada a confiabilidade

\end{easylist}	


\section{Processo de análise dos dados}

Análise dos dados será feita através de 4 etapas:

\begin{easylist}[itemize]	
	
	& \textbf{Categorização: } Organização dos dados em duas categorias - qualitativos e quantitativos. Os 		dados qualitativos referem-se aos questionários realizados. Os dados quantitativos, por sua vez, 			referem-se aos valores numéricos da solução de DW para monitoramento de métricas. 
	& \textbf{Exibição: } Consiste na organização dos dados coletados para serem exibidos através de 				gráficos, tabelas e texto para poderem ser analisados. 
	& \textbf{Verificação: } Atestar padrões, tendências e aspectos específicos dos significados dos 				dados. Procurando assim gerar uma discussão e interpretação de cada dado exibido.
	& \textbf{Conclusão: } Agrupamento dos resultados mais relevantes das discussões e interpretações dos 			dados anteriormente apresentados.
	
	\end{easylist}	


\section{Conclusão do capítulo}

\label{estudo de caso}

