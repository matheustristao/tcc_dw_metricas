\begin{resumo}

A qualidade do software depende da qualidade do código-fonte, um bom código-fonte é um bom indicador de qualidade interna do produto de \textit{software} \cite{ISO25023}. Portanto o monitoramento de métricas de código-fonte de um \textit{software} significa monitorar sua qualidade. Existem diversas soluções e ferramentas para se obter um monitoramento de métricas de \textit{software} e que conseguem extrair valores de métricas de código com facilidade, mas a decisão sobre o que fazer com os dados extraídos ainda esbarrem na dificuldade relacionada à visibilidade e interpretação dos dados. Neste trabalho buscou-se analisar a eficácia e eficiência do uso de um ambiente de \textit{Data Warehousing} para facilitar a interpretação, visibilidade e avaliação das métricas de código-fonte, associando-as a cenários de limpeza com o objetivo de apoiar as tomadas de decisão que reflitam na alteração do código-fonte na busca por um \textit{software }com mais qualidade. Este trabalho apresenta as fundamentações teóricas necessárias para o entendimento desta solução bem como elementos que dizem respeito a sua arquitetura e requisitos de negócio. Para realizar a pesquisa sobre sua eficácia e eficiência, foi elaborado um projeto para a realização de uma investigação empírica através da técnica do estudo de caso, que visa responder questões qualitativas e quantitativas a respeito da eficácia e eficiência na utilização da solução citada em um órgão público federal.

 \vspace{\onelineskip}
    
 \noindent
 \textbf{Palavras-chaves}: Métricas de Código-Fonte. \textit{Data Warehousing}. \textit{Data Warehouse}
\end{resumo}
